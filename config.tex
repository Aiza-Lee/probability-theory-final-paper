\usepackage{fontspec}      % 字体设置依赖

% --- 基础宏包 ---
\usepackage{geometry}       % 页面布局
\usepackage{fancyhdr}       % 页眉页脚
\usepackage{setspace}       % 行间距
\usepackage{titlesec}       % 标题样式定制
\usepackage{caption}        % 图表标题美化
\usepackage{booktabs}       % 专业三线表
\usepackage{float}          % 图片浮动控制
\usepackage{graphicx}       % 图片插入
\usepackage{listings}       % 代码块
\usepackage{xcolor}         % 颜色支持
\usepackage{subcaption}     % 子图支持

% --- 数学宏包 ---
\usepackage{amsmath}
\usepackage{amssymb}
\usepackage{amsthm}
\usepackage{bm}
\usepackage{mathrsfs}

% 让公式、图片、表格按章节编号 (例如: (1.1), 图 2.1)
\numberwithin{equation}{section}
\numberwithin{figure}{section}
\numberwithin{table}{section}

% --- 引用管理 (推荐使用 natbib 替代 cite) ---
\usepackage[numbers,sort&compress]{natbib} 

% --- 引用与超链接 (hyperref 通常建议最后加载) ---
\usepackage{hyperref}

\hypersetup{
    colorlinks=true, 
    linkcolor=blue, 
    citecolor=blue, 
    urlcolor=blue,
    pdftitle={概率论与数理统计课程小论文 - 蒙特卡洛积分和MIS在路径追踪中的应用},
    pdfauthor={李杰},
    pdfsubject={概率论期末论文},
    pdfkeywords={蒙特卡洛积分, 路径追踪, 多重重要性采样, MIS}
}

% --- 页面布局设置 ---
\geometry{left=2.5cm, right=2.5cm, top=2.5cm, bottom=2.5cm}
\setlength{\parskip}{0.5em} % 段间距
\setlength{\headheight}{14pt} % 修正 fancyhdr 警告
\onehalfspacing             % 1.5倍行距 (适合中文阅读)

% --- 页眉页脚设置 ---
\pagestyle{fancy}
\fancyhf{} % 清空当前设置
\fancyhead[C]{\small 概率论结课论文} % 页眉居中内容
\fancyfoot[C]{\thepage} % 页脚居中页码
\renewcommand{\headrulewidth}{0.5pt} % 页眉分割线宽度

% --- 图表标题美化 ---
\captionsetup{font={small,bf}, labelfont={bf}} 

% --- 概率论常用自定义命令 ---
\newcommand{\E}{\mathbf{E}}     % 期望
\renewcommand{\P}{\mathbb{P}}   % 概率
\newcommand{\Var}{\mathbf{D}}   % 方差
\newcommand{\Cov}{\mathrm{Cov}} % 协方差
\newcommand{\R}{\mathbb{R}}     % 实数集
\newcommand{\N}{\mathbb{N}}     % 自然数集
\newcommand{\ind}{\mathbb{I}}   % 指示函数
\newcommand{\diff}{\mathrm{d}}  % 积分微分符号

% --- 中文常用字号快捷命令 ---
% 用法:{\xiaosi 这里是小四号字}
\newcommand{\chuhao}{\fontsize{42pt}{63pt}\selectfont}    % 初号
\newcommand{\xiaochu}{\fontsize{36pt}{54pt}\selectfont}   % 小初
\newcommand{\yihao}{\fontsize{28pt}{42pt}\selectfont}     % 一号
\newcommand{\erhao}{\fontsize{21pt}{31.5pt}\selectfont}   % 二号
\newcommand{\xiaoer}{\fontsize{18pt}{27pt}\selectfont}    % 小二
\newcommand{\sanhao}{\fontsize{15.75pt}{24pt}\selectfont} % 三号
\newcommand{\xiaosan}{\fontsize{15pt}{22.5pt}\selectfont} % 小三
\newcommand{\sihao}{\fontsize{14pt}{21pt}\selectfont}     % 四号
\newcommand{\xiaosi}{\fontsize{12pt}{18pt}\selectfont}    % 小四
\newcommand{\wuhao}{\fontsize{10.5pt}{15.75pt}\selectfont}% 五号
\newcommand{\xiaowu}{\fontsize{9pt}{13.5pt}\selectfont}   % 小五
\newcommand{\liuhao}{\fontsize{7.875pt}{11.25pt}\selectfont}% 六号
\newcommand{\xiaoliu}{\fontsize{5.25pt}{7.5pt}\selectfont}  % 小六

% 设置衬线、非衬线、等宽字体(带字体可用性判断)
\IfFontExistsTF{Times New Roman}
  {\setmainfont{Times New Roman}}
  {\setmainfont{TeX Gyre Termes}}
\IfFontExistsTF{FiraCode Nerd Font}
  {\setmonofont{FiraCode Nerd Font}}
  {\setmonofont{Fira Code}}
\IfFontExistsTF{SimSun}
  {\setCJKmainfont[AutoFakeBold=true]{SimSun}}
  {\setCJKmainfont[AutoFakeBold=true]{Source Han Serif SC}}
\IfFontExistsTF{SimHei}
  {\setCJKsansfont[AutoFakeBold=true]{SimHei}}
  {\setCJKsansfont[AutoFakeBold=true]{Source Han Sans SC}}

% --- 定理环境设置 ---
\newtheoremstyle{thmstyle}
  {3pt}{3pt}              % 上下间距
  {\itshape}{}            % 正文字体
  {\bfseries}{.}          % 标题字体和标点
  {0.5em}{}               % 标题后间距
\theoremstyle{thmstyle}
\newtheorem{theorem}{定理}[section]
\newtheorem{lemma}[theorem]{引理}
\newtheorem{corollary}[theorem]{推论}

\theoremstyle{definition}
\newtheorem{definition}{定义}[section]
\newtheorem{example}{例}[section]
\newtheorem{remark}{注}[section]
