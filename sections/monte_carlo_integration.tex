\section{蒙特卡洛积分}

\subsection{定义}
蒙特卡洛积分是一种利用随机采样来估计积分值的数值方法。其基本思想是通过在积分区域内随机生成点,然后根据这些点的函数值来近似计算积分。\cite{FoCG}

形式化地,设我们要计算如下积分:
\begin{equation}
	I = \int_{x\in\Omega} f(x) \, \diff x .
\end{equation}

取独立同分布的随机样本 $X_1, X_2, \ldots, X_N$,样本服从概率密度函数 $p(x)$,则蒙特卡洛积分对$I$的估计为:

\begin{equation}
	\hat{I} = \frac{1}{N} \sum_{i=1}^N \frac{f(x_i)}{p(x_i)} .
\end{equation}

\subsection{无偏性证明}
$\hat{I}$的期望:
\begin{equation*}
	\begin{aligned}
		\E \hat{I}
		&= \E \left[ \frac{1}{N} \sum_{i=1}^N \frac{f(X_i)}{p(X_i)} \right] \\
		&= \frac{1}{N} \sum_{i=1}^N \E \left[ \frac{f(X_i)}{p(X_i)} \right] \\
		&= \E \left[ \frac{f(X_1)}{p(X_1)} \right] \\
		&= \int_{\Omega} \frac{f(x)}{p(x)} p(x) \, \diff x = \int_{\Omega} f(x) \, \diff x = I .
	\end{aligned}
\end{equation*}

\subsection{最小化方差分析}
$\hat{I}$的方差:
\begin{equation}
	\begin{aligned}
		\Var \hat{I}
		&= \Var \left[ \frac{1}{N} \sum_{i=1}^N \frac{f(X_i)}{p(X_i)} \right] \\
		&= \frac{1}{N^2} \sum_{i=1}^N \Var \left[ \frac{f(X_i)}{p(X_i)} \right] \\
		&= \frac{1}{N} \Var \left[ \frac{f(X_1)}{p(X_1)} \right] \\
		&= \frac{1}{N} \left( \E \left[ \frac{f(X_1)}{p(X_1)} \right]^2 - I^2 \right) \\
		&= \frac{1}{N} \left( \int_\Omega \frac{f(x)^2}{p(x)} \diff x - I^2 \right).
	\end{aligned}
	\label{eq:mc-variance}
\end{equation}

显然,当采样次数$N$增加,方差会减小,对应渲染结果的噪点会更少。
再看括号内部,当被积函数 $f(x)$ 在某些区域值较大时,如果采样分布 $p(x)$ 在这些区域的概率较小,则会导致方差增大。

因此,选择合适的采样分布对于降低方差至关重要。
更加具体地,使用拉格朗日乘子法\cite{Shuxueguihua2012}分析怎样的概率分布函数(采样策略)可以最小化方差。

问题描述:
\begin{equation}
	\begin{aligned}
		\min_{p(x)} N \cdot \Var \hat{I} \\ 
		\text{s.t.} \quad
		\int_{\Omega} p(x) \diff x - 1 = 0 &\quad \\
		-p(x) \le 0 &\quad.
	\end{aligned}
\end{equation}
构造拉格朗日函数:
\begin{equation}
	\mathcal{L}(p(x), \lambda, \mu) = 
		\int_\Omega \frac{f(x)^2}{p(x)} \diff x - I^2 
		+ \lambda \left( \int_{\Omega} p(x) \diff x - 1 \right) 
		+ \mu (-p(x)).
\end{equation}
由KKT条件中的驻点条件,对 $p(x)$ 求变微分并令其为零:
\begin{equation*}
	\frac{\delta\mathcal{L}}{\delta p(x)} = -\frac{f(x)^2}{p(x)^2} + \lambda - \mu = 0 .
\end{equation*}
化简后得到:
\begin{equation*}
	p(x) = \frac{|f(x)|}{\sqrt{\lambda - \mu}} .
\end{equation*}
由KKT条件中的可行性条件,即归一化条件,可以得到最优的概率分布为:
\begin{equation}
	p^*(x) = \frac{|f(x)|}{\int_{\Omega} |f(x)| \diff x} .
\end{equation}

这表明,为了最小化蒙特卡洛积分的方差,采样分布应与被积函数的绝对值成正比,这也是后面MIS的理论依据。