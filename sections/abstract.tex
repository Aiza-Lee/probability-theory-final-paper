\begin{abstract}

在光线追踪中,通过辐射度量学,我们可以将光传输的过程建模为积分方程,对于复杂场景下的光照计算,解析解往往难以获得。
蒙特卡洛积分作为一种数值积分方法,通过随机采样的方式对积分结果做无偏估计,广泛应用于路径追踪算法中以模拟全局光照效果。
然而,单一的采样策略在某些边缘情况下可能会导致估计方差过大,呈现在渲染结果上即为噪点。
为了解决这一问题,多重重要性采样(Multiple Importance Sampling, MIS)被提出,通过多种采样方式的结合,有效降低了估计的方差,在有限的运算量中,往往对渲染结果的质量有很大的提升。
本文将简单介绍蒙特卡洛积分的基本原理,通过cpu软光追复现路径追踪算法,并在此基础上应用多重重要性采样技术,展示其在实际渲染中的效果提升。

\vspace{1em}
\noindent{\xiaosi
	\textbf{关键词:} 概率论,路径追踪,蒙特卡洛积分,计算机图形学,MIS(Multiple Importance Sampling),渲染。}

\end{abstract}