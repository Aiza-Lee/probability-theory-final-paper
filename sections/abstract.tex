\begin{abstract}

在路径追踪中,通过辐射度量学,我们可以将光传输的过程建模为积分方程,对于复杂场景往往难以获得解析解。
蒙特卡洛积分作为一种数值积分方法,通过随机采样的方式对积分结果做无偏估计,广泛应用于路径追踪算法中以模拟全局光照效果。
同时,不同的随机采样方式(不同的概率分布函数)会导致不同的方差,本文通过拉格朗日乘子法求解最优的概率分布函数。

单一的采样策略在复杂场景中常表现为高方差,在渲染结果上呈现为噪点。
本文将介绍多重重要性采样(Multiple Importance Sampling, MIS)技术,
通过结合多种采样策略,显著降低估计的方差,提高渲染质量和效率。

本文详细阐述了蒙特卡洛积分的基本原理,并通过基于CPU的路径追踪渲染器复现了相关算法。
结果表明,逼近最优分布的采样策略能显著提高渲染质量;
而在复杂场景下,多重重要性采样技术在同等采样次数下展现出了更优的鲁棒性和收敛速度。

\vspace{1em}
\noindent{\xiaosi
	\textbf{关键词:}
	概率论,蒙特卡洛积分,计算机图形学,路径追踪,双向反射分布函数,多重重要性采样,渲染。
}

\end{abstract}