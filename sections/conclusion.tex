\section{总结}

本文探讨了蒙特卡洛积分及其在计算机图形学路径追踪算法中的应用。
从概率论的基础理论出发,详细阐述了蒙特卡洛积分的定义、无偏性证明及其方差特性。
通过引入拉格朗日乘子法,从理论上推导出了最小化方差的最优采样概率分布,即采样概率密度函数应与被积函数的绝对值成正比。

在路径追踪的具体实践中,分析了渲染方程的积分形式,并指出单一采样策略在面对复杂场景时的局限性。
例如,对于光滑表面或强光源场景,单一的BRDF采样或光源采样往往难以同时保持低方差,导致渲染图像出现明显的噪点。
为此,重点介绍了多重重要性采样(MIS)技术。
通过构建加权组合的估计量,MIS 能够有效地结合多种采样策略的优势。
通过严格的理论分析,证明了 MIS 估计量的无偏性,并推导得出其方差上界不高于任意单一采样策略的方差,从而在理论上保证了该方法的鲁棒性。

通过基于 CPU 的路径追踪渲染器的实现与对比实验,直观地验证了上述理论分析。
实验结果表明,相比于均匀采样或单一的余弦加权采样,逼近最优分布的采样策略能显著减少噪点。
特别是在复杂光照条件下,MIS 技术展现出了卓越的性能,它能够在同等采样次数下大幅降低估计方差,提升渲染画面的质量与收敛速度。

从理论推导到算法实现,再到实验验证,展示了概率论在计算机图形学中的重要地位与应用价值。