\section{多重重要性采样(MIS)}

\subsection{简介}
在上一节的末尾,展示了不同采样次数下的路径追踪渲染结果(见图~\ref{fig:sampling-comparison})。
但是即便采样次数增加到每像素1000次,渲染结果中仍然存在明显的噪点。

在大多数情况下,蒙特卡洛积分中的被积函数的BRDF和$L_i(\omega_i)$项都是非常复杂的,难以直接找到最优分布进行采样。
单一的采样策略在某些边缘情况下可能会导致估计方差过大(在前文对最优分布的分析中有阐述),呈现在渲染结果上即为噪点。
为了解决这一问题,多重重要性采样(Multiple Importance Sampling, MIS)被提出——通过多种采样方式的结合,
有效降低了估计的方差,在有限的运算量中,往往对渲染结果的质量有很大的提升\cite{PBRT4}。

\subsection{核心思想}
基于蒙特卡洛积分,要解决的问题仍然是估计:
\begin{equation}
	I = \int_\Omega f(x) \diff x.
\end{equation}

但是现在,有多种采样方式可供选择,假设有$m$种采样方式,概率分布分别为$p_i(x), i=1,2,\ldots,m$。

对于每一种采样方式,可以独立地进行采样,得到$n_i$个样本点$x_{i,1}, x_{i,2}, \ldots, x_{i,n_i}$。
总的样本数为$N = \sum_{i=1}^m n_i$。

定义权重函数为:
\begin{equation}
	w_i(x) = \frac{n_i p_i(x)}{\sum_{j=1}^m n_j p_j(x)}.
	\label{eq:mis-weight-function}
\end{equation}

记:
\begin{equation}
	S(x) = \sum_{j=1}^m n_j p_j(x).
	\label{eq:mis-sum-pdf}
\end{equation}

至此,可以构造多重重要性采样的估计量\cite{Veach1995}为:
\begin{equation}
	\hat{I}_\text{MIS} = \sum_{i=1}^m \frac{1}{n_i} \sum_{k=1}^{n_i} \frac{f(x_{i,k}) w_i(x_{i,k})}{p_i(x_{i,k})}.
\end{equation}

同时可以得到 MIS 的混合采样分布为:
\begin{equation}
	p_\text{MIS}(x) = \frac{S(x)}{N} = \sum_{j=1}^m \frac{n_j}{N} p_j(x).
\end{equation}

\subsection{无偏性证明}
多重重要性采样的估计量$\hat{I}_\text{MIS}$是$I$的无偏估计。证明如下:
\begin{equation}
	\begin{aligned}
		\E \hat{I}_\text{MIS}
		&= \E \left[ \sum_{i=1}^m \frac{1}{n_i} \sum_{k=1}^{n_i} \frac{f(X_{i,k}) w_i(X_{i,k})}{p_i(X_{i,k})} \right] \\
		&= \sum_{i=1}^m \E \left[ \frac{f(X_{i,1}) w_i(X_{i,1})}{p_i(X_{i,1})} \right] \\
		&= \sum_{i=1}^m \int_\Omega \frac{f(x) w_i(x)}{p_i(x)} p_i(x) \diff x \\
		&= \sum_{i=1}^m \int_\Omega f(x) w_i(x) \diff x \\
		&= \int_\Omega f(x) \left( \sum_{i=1}^m w_i(x) \right) \diff x \\
		&= \int_\Omega f(x) \diff x = I.
	\end{aligned}
\end{equation}

可以看出,该估计量的无偏性依赖于权重函数\eqref{eq:mis-weight-function} 的归一化性质,即$\sum_{i=1}^m w_i(x) = 1$。

\subsection{方差分析}
多重重要性采样估计量$\hat{I}_\text{MIS}$的方差:
\begin{equation}
	\begin{aligned}
		\Var \hat{I}_\text{MIS}
		&= \Var \left[ \sum_{i=1}^m \frac{1}{n_i} \sum_{k=1}^{n_i} \frac{f(X_{i,k}) w_i(X_{i,k})}{p_i(X_{i,k})} \right] \\
		&= \sum_{i=1}^m \Var \left[ \frac{1}{n_i} \sum_{k=1}^{n_i} \frac{f(X_{i,k}) w_i(X_{i,k})}{p_i(X_{i,k})} \right] \\
		&= \sum_{i=1}^m \frac{1}{n_i} \Var \left[ \frac{f(X_{i,1}) w_i(X_{i,1})}{p_i(X_{i,1})} \right] \\
		&= \sum_{i=1}^m \frac{1}{n_i} \left( \E \left[ \frac{f(X_{i,1}) w_i(X_{i,1})}{p_i(X_{i,1})} \right]^2 - I^2 \right) \\
		&= \sum_{i=1}^m \frac{1}{n_i} \left( \int_\Omega \frac{f(x)^2 w_i(x)^2}{p_i(x)} \diff x - I^2 \right).
	\end{aligned}
\end{equation}

带入式~\eqref{eq:mis-sum-pdf},可以得到:
\begin{equation}
	\Var \hat{I}_\text{MIS}
	= \int_\Omega f(x)^2 \frac{1}{S(x)} \diff x - \left( \sum_{i=1}^m \frac{1}{n_i} \right) I^2.
	\label{eq:mis-variance-simplified}
\end{equation}

考察式中$\frac{1}{S(x)}$:
\begin{equation}
	\frac{1}{S(x)} = \frac{1}{\sum_{j=1}^m n_j p_j(x)} \le \min_{1\le i \le m} \frac{1}{n_i p_i(x)}.
	\label{eq:mis-variance-bound}
\end{equation}

将不等式~\eqref{eq:mis-variance-bound}带入式~\eqref{eq:mis-variance-simplified},可以得到:
\begin{equation*}
	\Var \hat{I}_\text{MIS} \le \int_\Omega f(x)^2 \min_{1\le i \le m}\left\{ \frac{1}{n_i p_i(x)} \right\} \diff x - \left( \sum_{i=1}^m \frac{1}{n_i} \right) I^2.
\end{equation*}

进一步得到,对于任意单一采样策略$i$,对照单一采样策略方差表达式~\eqref{eq:mc-variance},都有:
\begin{equation}
	\Var \hat{I}_\text{MIS} \le \int_\Omega f(x)^2 \frac{1}{n_i p_i(x)} \diff x - \frac{1}{n_i} I^2 = \Var \hat{I}_i.
\end{equation}

也即,多重重要性采样的方差不大于任意单一采样策略的方差,证明该估计在减小方差方面的有效性。

同时,对于多重重要性采样的方差最小化问题,可以通过调整各采样策略的样本数量$n_i$来实现。
该问题的具体分析较为复杂,这里不做展开。在工程实践中,通常会根据经验或预先的分析结果来分配各采样策略的样本数量。

\subsection{比较多重重要性采样和单一采样策略的方差}

MIS 在路径追踪中最常见的应用场景是结合光源采样和BRDF采样两种策略。

在~\eqref{eq:mc-variance}的分析中,我们已经知道,$f(x)$在某些区域值较大时,如果采样分布$p(x)$在这些区域的概率较小,则会导致方差增大。
在实际渲染过程中,$f(x)$较大的区域通常对应于光源方向附近(入射光照$L_i$较大)以及BRDF函数值较大的方向(例如高光反射方向)。
因此,单一采样策略往往难以同时兼顾这两种情况,从而导致较大的方差。
而MIS通过结合多种采样策略,可以更好地覆盖$f(x)$较大的区域,从而有效降低方差。

这里将展示结合光源采样和BRDF采样的MIS策略与单一采样策略在渲染效果上的差异。

\begin{figure}[H]
	\centering
	\begin{subfigure}[b]{0.45\textwidth}
		\centering
		\includegraphics[width=\textwidth]{pictures/misvssingle/mis_comparison_1_mis.png}
		\caption{MIS混合采样策略}
		\label{fig:mis-vs-single-comparison-1-mis}
	\end{subfigure} \\
	\begin{subfigure}[b]{0.45\textwidth}
		\centering
		\includegraphics[width=\textwidth]{pictures/misvssingle/mis_comparison_2_light.png}
		\caption{单一光源采样策略}
		\label{fig:mis-vs-single-comparison-2-light}
	\end{subfigure}
	\begin{subfigure}[b]{0.45\textwidth}
		\centering
		\includegraphics[width=\textwidth]{pictures/misvssingle/mis_comparison_3_material.png}
		\caption{单一BRDF采样策略}
		\label{fig:mis-vs-single-comparison-3-material}
	\end{subfigure}
	\caption{不同采样策略下的渲染结果对比(上:MIS,左下:光源采样,右下:BRDF采样)}
	\label{fig:mis-vs-single-comparison-1}
\end{figure}
该图是对Veach论文\cite{Veach1995}中MIS效果的复现实验,整个场景有微弱的环境光照,和五个大小,颜色互异的球状光源,
在下方是四面模糊度不同的长方形镜子。

从图~\ref{fig:mis-vs-single-comparison-1}中可以明显看出,MIS策略的渲染结果噪点更少,质量更高。

相应的,从光源直接采样的单一光源策略在对红色的小光源的采样上效果较好,但在大面积光源的采样上,
会因为从光源采样的方向和镜面反射的方向不匹配导致噪点较多。
按照材质的性质进行BRDF采样的单一材质策略在对大面积光源的采样上效果较好,因为随机反射的方向大概率会落在光源区域,
同理对小光源的采样效果较差。

降低渲染的方差的作用不仅仅在与提高图片质量,还在于可以减少每个像素的采样次数,从而提高渲染效率。
下面会展示在相同采样次数下,MIS和单一采样策略的渲染效果的明显差异。
\begin{figure}[H]
	\centering
	\begin{subfigure}[b]{0.45\textwidth}
		\centering
		\includegraphics[width=\textwidth]{pictures/misvssingle/playground_mis.png}
		\caption{MIS策略,每像素采样100次}
		\label{fig:mis-vs-single-comparison-playground-mis}
	\end{subfigure} \\
	\begin{subfigure}[b]{0.45\textwidth}
		\centering
		\includegraphics[width=\textwidth]{pictures/misvssingle/playground_light.png}
		\caption{单一光源策略,每像素采样100次}
		\label{fig:mis-vs-single-comparison-playground-light}
	\end{subfigure}
	\begin{subfigure}[b]{0.45\textwidth}
		\centering
		\includegraphics[width=\textwidth]{pictures/misvssingle/playground_material.png}
		\caption{单一BRDF策略,每像素采样100次}
		\label{fig:mis-vs-single-comparison-playground-material}
	\end{subfigure}
	\caption{不同采样策略下的渲染效率对比(上:MIS,左下:光源采样,右下:BRDF采样)}
	\label{fig:mis-vs-single-comparison-playground}
\end{figure}

从图~\ref{fig:mis-vs-single-comparison-playground}中可以明显看出,在相同采样次数下,MIS策略的渲染结果质量显著优于单一采样策略,
噪点更少,细节更清晰。
这进一步验证了MIS在降低方差和提高渲染效率方面的优势。