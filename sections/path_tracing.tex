\section{光线追踪和路径追踪}

在图形学渲染领域,光线追踪是一种基于蒙特卡洛积分的,模拟场景中光线传播和能量传递、损失,并最终渲染为图像的技术。

\begin{figure}[H]
	\centering
	\includegraphics[width=0.7\textwidth]{pictures/光线追踪示意图.png}
	\caption{光线追踪示意图(截图引自 GAMES101 公开课\cite{GAMES101})}
	\label{fig:games101-raytracing}
\end{figure}

\subsection{BRDF(双向反射分布函数)}
基于辐射度量学\cite{GAMES101},我们在物理世界上抽象出 BRDF,来描述光线的传播。

BRDF。表示从某个微分的立体角接受的光照会如何映射到发出(反射)光照的方向和强度。
用数学语言描述,对于位于$p$的面积微元上,定义双向反射分布函数:
\begin{equation}
	f(\omega_i \rightarrow \omega_r) := \frac{\mathrm d L_r(w_r)}{\mathrm d E(\omega_i)}
 = \frac{\mathrm d L_r(w_r)}{L_i(\omega_i)\cos\theta_i\mathrm d \omega_i}.
\end{equation}
\begin{figure}[H]
	\centering
	\includegraphics[width=0.6\textwidth]{pictures/BRDF图示.png}
	\caption{BRDF示意图(截图引自 GAMES101 公开课\cite{GAMES101})}
	\label{fig:games101-brdf}
\end{figure}
进而可以将某一点$p$在某个立体角方向$\omega_r$上的辐照度$L_r$表示为:
\begin{equation}
	L_r(\omega_r) = L_e(\omega_r) 
		+ \int_{H^+} f(\omega_i \rightarrow \omega_r)
					L_i(\omega_i)\cos \theta_i
					\diff \omega_i.
\end{equation}
(注:公式中省略了表示该反射点的坐标的参数$p$)

等式右侧的复杂积分式可以通过蒙特卡洛积分求解,关键在于如何选择合适的概率分布函数,进而选择何时的采样的立体角方向。
