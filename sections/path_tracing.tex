\section{路径追踪}

\subsection{简介}
在图形学渲染领域,路径追踪是一种基于蒙特卡洛积分的,模拟场景中光线传播和能量传递、损失,并最终渲染为图像的技术。

\begin{figure}[H]
	\centering
	\includegraphics[width=0.7\textwidth]{pictures/路径追踪示意图.png}
	\caption{路径追踪示意图(截图引自 GAMES101 公开课\cite{GAMES101})}
	\label{fig:games101-raytracing}
\end{figure}

最终需要渲染的是图~\ref{fig:games101-raytracing}中 “image plane” 上的每个像素点。

每个反射点按照一定概率反射一条光线,在场景中随机反射得到一个颜色。
对于每个像素点,从相机发出多条光线,并对每条光线计算的颜色取平均。这就是路径追踪的基本思路。
蒙特卡洛积分的作用在于:每次随机反射的光线,都会对应一定的概率密度,从而可以通过蒙特卡洛积分来估计该光线的最终颜色。

对于各种无偏性的采样,采样效果由估计值的方差决定,方差越小,在有限采样次数下,
得到的光线的结果就会更加接近真实值,从而渲染效果更好。
同时,较大的方差会体现为图片上的“噪点”。

\subsection{BRDF(双向反射分布函数)}
基于辐射度量学,我们在物理世界上抽象出 BRDF,来描述光线的传播\cite{GAMES101}。

BRDF。表示从某个微分的立体角接受的光照会如何映射到发出(反射)光照的方向和强度。
用数学语言描述,对于位于$p$的有向面积微元上,定义双向反射分布函数:
\begin{equation}
	f(\omega_i \rightarrow \omega_r) := \frac{\diff L_r(\omega_r)}{\diff E(\omega_i)}
 = \frac{\diff L_r(\omega_r)}{L_i(\omega_i)\cos\theta_i\diff \omega_i}.
\end{equation}
\begin{figure}[H]
	\centering
	\includegraphics[width=0.6\textwidth]{pictures/BRDF图示.png}
	\caption{BRDF示意图(截图引自 GAMES101 公开课\cite{GAMES101})}
	\label{fig:games101-brdf}
\end{figure}
进而可以将某一点$p$在某个立体角方向$\omega_r$上的辐照度$L_r$表示为:
\begin{equation}
	L_r(\omega_r) = L_e(\omega_r) 
		+ \int_{H^+} f(\omega_i \rightarrow \omega_r)
					L_i(\omega_i)\cos \theta_i
					\diff \omega_i.
\end{equation}
(注:公式中省略了表示该反射点的坐标的参数$p$)

等式右侧的复杂积分式可以通过蒙特卡洛积分求解,关键在于如何选择合适的概率分布函数,进而选择合适的采样的立体角方向。

\subsection{漫反射材质下不同采样策略的比较}

\begin{figure}[H]
	\centering
	\includegraphics[width=0.6\textwidth]{pictures/漫反射brdf.png}
	\caption{漫反射BRDF(截图引自 GAMES101 公开课\cite{GAMES101})}
	\label{fig:lambertian-brdf}
\end{figure}

漫反射BRDF(朗伯反射),其BRDF函数值为常数,如图~\ref{fig:lambertian-brdf}所示。
本节使用该材质展示不同的采样次数对最终渲染效果的影响。

策略一:在上半球上随机采样。

策略二:按照入射光照立体角方向和平面夹角的余弦值$\cos\theta_i$进行采样。

实际渲染效果:

\begin{figure}[H]
	\centering
	\begin{subfigure}[b]{0.45\textwidth}
		\centering
		\includegraphics[width=\textwidth]{pictures/uniformvscosine/comparison_sampling_16.png}
		\caption{每像素随机采样16次}
		\label{fig:comparison_sampling_16}
	\end{subfigure}
	\begin{subfigure}[b]{0.45\textwidth}
		\centering
		\includegraphics[width=\textwidth]{pictures/uniformvscosine/comparison_sampling_50.png}
		\caption{每像素随机采样50次}
		\label{fig:comparison_sampling_50}
	\end{subfigure}
	\begin{subfigure}[b]{0.45\textwidth}
		\centering
		\includegraphics[width=\textwidth]{pictures/uniformvscosine/comparison_sampling_100.png}
		\caption{每像素随机采样100次}
		\label{fig:comparison_sampling_100}
	\end{subfigure}
	\begin{subfigure}[b]{0.45\textwidth}
		\centering
		\includegraphics[width=\textwidth]{pictures/uniformvscosine/comparison_sampling_1000.png}
		\caption{每像素随机采样1000次}
		\label{fig:comparison_sampling_1000}
	\end{subfigure}
	\caption{不同采样策略下的渲染结果对比(左:半球上随机,右:按照被积函数采样)}
	\label{fig:sampling-comparison}
\end{figure}

从图~\ref{fig:sampling-comparison}中可以明显看出,随着每像素采样次数的增加,
噪点逐渐减少,渲染效果更好,无偏性和方差减小得到验证。

对于两种采样策略的差异,体现得并不明显,原因在于被积函数具体的形状是未知的,两种采样策略在最终效果上差异不大,
但还是可以观察到,在球的上顶点附近光照强度较大的地方,使用按照$\cos\theta_i$采样的效果更好一些,
因为该采样策略更倾向于在该区域采样,从而降低了方差。