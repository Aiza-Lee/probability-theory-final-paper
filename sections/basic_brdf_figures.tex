\subsection{比较随机采样和按照被积函数采样}

\begin{figure}[H]
	\centering
	\includegraphics{pictures/漫反射brdf.png}
	\caption{漫反射BRDF(截图引自 GAMES101 公开课\cite{GAMES101})}
	\label{fig:lambertian-brdf}
\end{figure}

漫反射BRDF(朗伯反射),其BRDF函数值为常数,如图~\ref{fig:lambertian-brdf}所示。
本节使用该材质展示不同的采样策略对最终渲染效果的影响。

策略一:在上半球上随机采样。

策略二:按照入射光照立体角方向和平面夹角的余弦值进行采样
(即按照被积函数$(\text{BRDF}\cdot L_i \cos\theta_i$)进行采样)。

预期的渲染效果差异:
两种策略的不同不影响蒙特卡洛积分本身的无偏性,但是在方差上,策略一会比策略二的更大,
渲染结果体现为更多的噪点。

实际渲染效果差异:

\begin{figure}[H]
	\centering
	\begin{subfigure}[b]{0.45\textwidth}
		\centering
		\includegraphics[width=\textwidth]{pictures/uniformvscosine/comparison_sampling_16.png}
		\caption{每像素随机采样16次}
		\label{fig:comparison_sampling_16}
	\end{subfigure}
	\begin{subfigure}[b]{0.45\textwidth}
		\centering
		\includegraphics[width=\textwidth]{pictures/uniformvscosine/comparison_sampling_50.png}
		\caption{每像素随机采样50次}
		\label{fig:comparison_sampling_50}
	\end{subfigure}
	\begin{subfigure}[b]{0.45\textwidth}
		\centering
		\includegraphics[width=\textwidth]{pictures/uniformvscosine/comparison_sampling_100.png}
		\caption{每像素随机采样100次}
		\label{fig:comparison_sampling_100}
	\end{subfigure}
	\begin{subfigure}[b]{0.45\textwidth}
		\centering
		\includegraphics[width=\textwidth]{pictures/uniformvscosine/comparison_sampling_1000.png}
		\caption{每像素随机采样1000次}
		\label{fig:comparison_sampling_1000}
	\end{subfigure}
	\caption{不同采样策略下的渲染结果对比(左:半球上随机,右:按照被积函数采样)}
	\label{fig:sampling-comparison}
\end{figure}

从图~\ref{fig:sampling-comparison}中可以明显看出,随着每像素采样次数的增加,
噪点逐渐减少,渲染效果更好,无偏性得到验证。

对于两种采样策略的差异,体现得并不明显,在图~\ref{fig:comparison_sampling_50}和~\ref{fig:comparison_sampling_100}中会相对明显,
可以观察到采用按照被积函数分布的采样策略,得到的图像的噪点较少。