\subsection{漫反射材质下不同采样策略的比较}

\begin{figure}[H]
	\centering
	\includegraphics[width=0.6\textwidth]{pictures/漫反射brdf.png}
	\caption{漫反射BRDF(截图引自 GAMES101 公开课\cite{GAMES101})}
	\label{fig:lambertian-brdf}
\end{figure}

漫反射BRDF(朗伯反射),其BRDF函数值为常数,如图~\ref{fig:lambertian-brdf}所示。
本节使用该材质展示不同的采样次数对最终渲染效果的影响。

策略一:在上半球上随机采样。

策略二:按照入射光照立体角方向和平面夹角的余弦值$\cos\theta_i$进行采样。

实际渲染效果:

\begin{figure}[H]
	\centering
	\begin{subfigure}[b]{0.45\textwidth}
		\centering
		\includegraphics[width=\textwidth]{pictures/uniformvscosine/comparison_sampling_16.png}
		\caption{每像素随机采样16次}
		\label{fig:comparison_sampling_16}
	\end{subfigure}
	\begin{subfigure}[b]{0.45\textwidth}
		\centering
		\includegraphics[width=\textwidth]{pictures/uniformvscosine/comparison_sampling_50.png}
		\caption{每像素随机采样50次}
		\label{fig:comparison_sampling_50}
	\end{subfigure}
	\begin{subfigure}[b]{0.45\textwidth}
		\centering
		\includegraphics[width=\textwidth]{pictures/uniformvscosine/comparison_sampling_100.png}
		\caption{每像素随机采样100次}
		\label{fig:comparison_sampling_100}
	\end{subfigure}
	\begin{subfigure}[b]{0.45\textwidth}
		\centering
		\includegraphics[width=\textwidth]{pictures/uniformvscosine/comparison_sampling_1000.png}
		\caption{每像素随机采样1000次}
		\label{fig:comparison_sampling_1000}
	\end{subfigure}
	\caption{不同采样策略下的渲染结果对比(左:半球上随机,右:按照被积函数采样)}
	\label{fig:sampling-comparison}
\end{figure}

从图~\ref{fig:sampling-comparison}中可以明显看出,随着每像素采样次数的增加,
噪点逐渐减少,渲染效果更好,无偏性和方差减小得到验证。

对于两种采样策略的差异,体现得并不明显,原因在于被积函数具体的形状是未知的,两种采样策略在最终效果上差异不大,
但还是可以观察到,在球的上顶点附近光照强度较大的地方,使用按照$\cos\theta_i$采样的效果更好一些,
因为该采样策略更倾向于在该区域采样,从而降低了方差。